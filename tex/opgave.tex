\section{Et kort oprids af dansk handicappolitik}

I den følgende synopsis vil jeg diskutere nedslag i en policy-analyse af dansk handicappolitik, som den fremstår i dag.
Jeg strukturerer analysen med afsæt i nogle af de forskellige interessenter på området:

- De handicappede, som (burde være) omdrejningspunktet for det hele
- Myndighederne, som indehaver af definitionsmagt, både hvad angår 'problemets' art; og hvordan det skal/kan/bør 'løses'
- De professionelle i feltet, som udøvere af velfærdsstatens gerning - og hvordan de udrustes til opgaven

Jeg vil teoretisk tage afsæt i \citetitle{goffmanAsylumsEssaysSocial1991} af Erving Goffman (\citeyear{goffmanAsylumsEssaysSocial1991}) for at belyse de 'indsatte' og 'ansatte' på handicapområdet.
For en forståelse af, hvordan 'de handicappede' er blevet et samfundsmæssigt anliggende, og hvordan dette anliggende imødekommes, vil jeg henvende mig til [[blumer?? Scheurer?? bægge??].
I gennemgangen af den moderne specialpædagogiske professions tilblivelse vil jeg støtte mig op ad bl. a. ???[[jakob ditlev bøje??]]
