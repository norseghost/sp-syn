\section{Handicappolitik i Danmark}

Jeg formoder, at jeg vil tage fat i den specialpædagogiske professions selvforståelse i mit speciale.
'Specialpædagogik' er forbundet med arbejdet med \textit{mennesker med nedsat fysisk og/eller kognitiv funktionsevne}\todo{find det præcise citat - og måske en kilde til forbindelsen}, som det så fint hedder i \citetitle{borne-ogsocialministerietBekendtgorelseAfLov2018}
Jeg vil dog forholdsvis konsekvent holde mig til 'handicappede' fremfor den mere forsigtige formulering ovenfor.
Både fordi, 'handicap' benyttes i samfundet i almindelighed (og også af 'de handicappede' specifikt); men også fordi, der er en (overvejende) mere relationel forankring i handicapbegrebet i dansk sammenhæng.\todo{Uddybes? evt med et oprids af forskellige handicapforståelser?}

I den følgende synopsis vil jeg diskutere nedslag i en policy-analyse af dansk handicappolitik, som den fremstår i dag.

Analysens formål er, en forståelse af, hvordan 'de handicappede' er blevet et samfundsmæssigt anliggende, og hvordan dette anliggende imødekommes.
\todo{PROBLEM FORMULERING}
Min undersøgelse er på baggrund af \citetitle{borne-ogsocialministerietBekendtgorelseAfLov2018}, samt \citetitle{klStyringAfDet2017}, en vejledning til det specialiserede voksenområde udarbejdet af Kommunernes Landssammenslutning i samarbejde med Børne- og socialministeriet, Økonomi- og indenrigsministeriet samt Finansministeriet (\cite{borne-ogsocialministerietBekendtgorelseAfLov2018}, \cite{klStyringAfDet2017}.
 Metodisk vil jeg henvende mig til \citeauthor{scheurichPolicyArchaeologyNew1994}.
Med afsæt i \citetitle{scheurichPolicyArchaeologyNew1994}, vil jeg udforske det handicappolitiske landskab i dagens Danmark.

I \citetitle{scheurichPolicyArchaeologyNew1994} anfægter han selvfølgeligheden af de “sociale problemer” som et empirisk fænomen \autocite{scheurichPolicyArchaeologyNew1994}.
Ved at tage for givet, at noget er et \textbf{problem}, tager man også for givet, at der er en \textbf{løsning}, der kan sikre samfundet stabilitet og videreførelse.
I stedet for bør og skal de såkaldte “sociale problemer” undersøges som de sociale konstruktioner de er.
Dermed eftersøges en policy-analyse der går ud over det historiske, evaluerende, og/eller komparative.
\citeauthor{scheurichPolicyArchaeologyNew1994} præsenterer policy-arkæologien som en alternativ analysemodel til de (post)positiviske analysemodeller.\todo{Eh. Tåler en omskrivning}

Ved at se på \cite[s. 300]{scheurichPolicyArchaeologyNew1994}:
\begin{enumerate}
  \item
    den sociale konstruktion af specifikke sociale problemer
  \item
    identifikation af sociale regelmæssigheder på tværs af problemdefinitioner
  \item
    den sociale konstruktion af acceptable løsningsmuligheder
  \item
    policy-analysens sociale formål
\end{enumerate}


\section{Handicappede i Danmark — eller: konstruktionen af et socialt problem}

\subsection{Et historisk oprids}
\todo{udenfor opgavens remit? måske holde sig til, hvad serviceloven/vejledning siger?}

Nutidens handicappolitikker kan spores tilbage til 1950erne\todo{Dato?? - Hur har en tidslinje (og er den god nok?}
Der kan godt nok synes lang vej fra nutidens idealer om 'et godt liv for alle' og 'deltagelse i samfundet'\todo{Kilde på disse idealer} men der var dog en grundlæggende idé, at samfundet efterhånden måtte træde til også for denne demografi.
\todo{TO SPOR - skoler vs bosteder -- uddybes?}
I løbet af 1960erne og '70erne var der dog strømninger, både internationalt og nationalt, der først betød, at handicappede fik stadfestet rettigheder ved lov i 1975; og en større overhaling af institutionerne på området i 1980.
Nu skulle de store totalinstitutioner afvikles; og de handicappede integreres i det danske samfund. 
i 1994 tilsluttede Danmark sig \citetitle{SalamancaErklaeringenOg1997}, der yderligere stadfæstede handicappedes ret til ligebehandling. Samme år gik de over til at få pension, og de handicappede fik nu også en personlig økonomi lige som “os andre”.
Denne normaliserings- og individualiseringslogik blev yderligere slået fast i 1998, hvor \citetitle{borne-ogsocialministerietBekendtgorelseAfLov2018} blev vedtaget.
Denne lov, godt tyve år senere, sætter et tydeligt individuelt fokus - både hvad gælder rettigheder og ansvar.

\subsection{Problemforståelsen i dansk handicappolitik i dag}
Det ovenstående peger på, en forskydning af \textit{problemet der skal løses} omkring de handicappede.
Fra at være til gene for omgivelserne, med en medicinsk afvigelsesforståelse (åndssvageanstalernes forstandere var læger)\todo{kilde} over mennesker, der aktivt skulle integreres og socialiseres, til \textit{individer} med rettigheder — og pligter — i og overfor samfundet.

\section{De sociale regelmæssigheder}

\citeauthor{scheurichPolicyArchaeologyNew1994} mener, at man kan se regelmæssigheder på tværs af sociale problemer.
Disse regelmæssigheder er med til at synliggøre noget som et “socialt fænomen”; samtidig med, at det også udgør de socialt troværdige løsningsmuligheder \cite[s. 301]{scheurichPolicyArchaeologyNew1994}
Hvilke regelmæssighed - det vil sige, forbindelser til det omkringliggende samfund — kan vi finde i de forskellige perspektiver nævnt ovenpå?

\begin{itemize}
  \item
    et individorienteret fokus, 
  \item
    hvor frihed udspringer af rettigheder
  \item
    men der er en medfølgende ansvarsfordring
  \item
    om at være produktiv
\end{itemize}

\section{Individet i fokus — eller: konstruktionen af acceptable løsningsforslag}

\section{Hvad (og hvem) gør denne analyse godt for?}
