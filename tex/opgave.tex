\section{Et kort oprids af dansk handicappolitik}

I den følgende synopsis vil jeg diskutere nedslag i en policy-analyse af dansk handicappolitik, som den fremstår i dag.
Jeg strukturerer analysen med afsæt i nogle af de forskellige interessenter på området:

- De handicappede, som (burde være) omdrejningspunktet for det hele
- Myndighederne, som indehaver af definitionsmagt, både hvad angår 'problemets' art; og hvordan det skal/kan/bør 'løses'
- De professionelle i feltet, som udøvere af velfærdsstatens gerning - og hvordan de udrustes til opgaven

Jeg vil teoretisk tage afsæt i \citetitle{goffmanAsylumsEssaysSocial1991} af Erving Goffman (\citeyear{goffmanAsylumsEssaysSocial1991}) for at belyse de 'indsatte' og 'ansatte' på handicapområdet.
For en forståelse af, hvordan 'de handicappede' er blevet et samfundsmæssigt anliggende, og hvordan dette anliggende imødekommes, vil jeg henvende mig til [[blumer?? Scheurer?? bægge??].
I gennemgangen af den moderne specialpædagogiske professions tilblivelse vil jeg støtte mig op ad bl. a. ???[[jakob ditlev bøje??]]

\subsection{Handicappede i Danmark}
Først, en begrebsafklaring: jeg bruger forholdsvis konsekvent 'handicappede' fremfor den mere forsigtige 'mennesker med nedsat fysisk og/eller psykisk funktionsevne'.
Både fordi, 'handicap' benyttes i samfundet i almindelighed (og af 'de handicappede' specifikt); men også fordi, der er en (overvejende) mere relationel forankring i handicapbegrebet i dansk sammenhæng. ((UDDYBES? DER ER IKKE MEGET PLADS I EN SYNOPSIS))  

\subsubsection{Et historisk oprids}
Nutidens handicappolitikker kan spores tilbage til 1950erne((TJEK DATO - Hur har en tidslinje)) ((OG TJEK VALIDITET I UDSAGNET))
Der kan godt nok synes lang vej fra nutidens idealer om 'et godt liv for alle' og 'deltagelse i samfundet'; men der var dog en grundlæggende idé, at samfundet efterhånden måtte træde til også for denne demografi.
((TO SPOR - skoler vs bosteder -- uddybes?))
I løbet af 1960erne og '70erne var der dog strømninger, både internationalt og nationalt, der først betød, at handicappede fik stædfestet rettigheder ved lov i 1975; og en større overhaling af institutionerne på området i 1980.
Nu skulle de store totalinstitutioner afvikles; og de handicappede integreres i det danske samfund. 
i 1994 tilsluttede Danmark sig \citetitle{SalamancaErklaeringenOg1997}, der yderligere stadfæstede handicappedes ret til ligebehandling. Samme år gik de over til at få pension, og de handicappede fik nu også en personlig økonomi lige som “os andre”.
Denne normaliserings- og individualiseringslogik blev yderligere slået fast i 1998, hvor \citetitle{borne-ogsocialministerietBekendtgorelseAfLov2018} blev vedtaget.
Denne lov, godt tyve år senere, sœtter et tydeligt individuelt fokus - både hvad gælder rettigheder og ansvar.

\subsubsection{De handicappedes interesseorganisationer}
Der er efterhånden dannet et større netværk af særorganisationer, der taler for de handicappede.
Disse er samlet i Danske Handicaporganisationer... (UDDYBES)

\subsection{De offentlige instanser}
Den markant invidividuerede forståelse i serviceloven er dermed også genspejlet i handicaporganisationernes selvbillede ((??? IKKE OGSÅ???)).
Det påhviler kommunerne at tildele støtte efter servicelovens bestemmelser, som beskrevet i fx §??? ((UDDRAG?? NÆPPE PLADS)).
Kommunerne er dog underlagt flere fordringer end, at sikre de handicappedes rettigheder, som bl.a. 'vejldeing til støtte til voksne' (KL og socialministeriet - Danske Regioner??)

\subsection{De professionelle}
Den velfærdsprofession, der er forbundet med arbejdet på det specialiserede socialområde, har siden ?? primært været pædagoger - indtil 199? \textbf{social}pædagoger; denne specialisering faldt bort ved indføringen af en generalistuddannelse
Senere er der dog tilnærminger mod en respecialisering af faget; i løbet af ??? reformer er der indført profiler og specialiseringer, der 
Denne udvikling skal også ses i lyset af, den ændrede organisationsform har 

\subsection{Problemforståelsen i dansk handicappolitik i dag}
Det ovenstående peger på, en
