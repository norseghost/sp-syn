\section{Et kort oprids af dansk handicappolitik}

I den følgende synopsis vil jeg diskutere nedslag i en policy-analyse af dansk handicappolitik, som den fremstår i dag.
Jeg strukturerer analysen med afsæt i nogle af de forskellige interessenter på området:

- De handicappede, som (burde være) omdrejningspunktet for det hele
- Myndighederne, som indehaver af definitionsmagt, både hvad angår 'problemets' art; og hvordan det skal/kan/bør 'løses'
- De professionelle i feltet, som udøvere af velfærdsstatens gerning - og hvordan de udrustes til opgaven

Jeg vil teoretisk tage afsæt i \citetitle{goffmanAsylumsEssaysSocial1991} af Erving Goffman (\citeyear{goffmanAsylumsEssaysSocial1991}) for at belyse de 'indsatte' og 'ansatte' på handicapområdet.
For en forståelse af, hvordan 'de handicappede' er blevet et samfundsmæssigt anliggende, og hvordan dette anliggende imødekommes, vil jeg henvende mig til [[blumer?? Scheurer?? bægge??].
I gennemgangen af den moderne specialpædagogiske professions tilblivelse vil jeg støtte mig op ad bl. a. ???[[jakob ditlev bøje??]]

\subsection{Handicappede i Danmark}
Først, en begrebsafklaring: jeg bruger forholdsvis konsekvent 'handicappede' fremfor den mere forsigtige 'mennesker med nedsat fysisk og/eller psykisk funktionsevne'.
Både fordi, 'handicap' benyttes i samfundet i almindelighed (og af 'de handicappede' specifikt); men også fordi, der er en (overvejende) mere relationel forankring i handicapbegrebet i dansk sammenhæng. ((UDDYBES? DER ER IKKE MEGET PLADS I EN SYNOPSIS))  

\subsubsection{Et historisk oprids}
Nutidens handicappolitikker kan spores tilbage til 1950erne((TJEK DATO - Hur har en tidslinje)) ((OG TJEK VALIDITET I UDSAGNET))
Der kan godt nok synes lang vej fra nutidens idealer om 'et godt liv for alle' og 'deltagelse i samfundet'; men der var dog en grundlæggende idé, at samfundet efterhånden måtte træde til også for denne demografi.
((TO SPOR - skoler vs bosteder -- uddybes?))
I løbet af 1960erne og '70erne var der dog strømninger, både internationalt og nationalt, der først betød, at handicappede fik stædfestet rettigheder ved lov i 1975; og en større overhaling af institutionerne på området i 1980.
Nu skulle de store totalinstitutioner afvikles; og de handicappede integreres i det danske samfund. 
i 1994 tilsluttede Danmark sig \citetitle{SalamancaErklaeringenOg1997}, der yderligere stadfæstede handicappedes ret til ligebehandling. Samme år gik de over til at få pension, og de handicappede fik nu også en personlig økonomi lige som “os andre”.
Denne normaliserings- og individualiseringslogik blev yderligere slået fast i 1998, hvor \citetitle{borne-ogsocialministerietBekendtgorelseAfLov2018} blev vedtaget.
Denne lov, godt tyve år senere, sœtter et tydeligt individuelt fokus - både hvad gælder rettigheder og ansvar.

\subsubsection{De handicappedes interesseorganisationer}
Der er efterhånden dannet et større netværk af særorganisationer, der taler for de handicappede.
Disse er samlet i Danske Handicaporganisationer... (UDDYBES)

\subsection{De offentlige instanser}
Den markant invidividuerede forståelse i serviceloven er dermed også genspejlet i handicaporganisationernes selvbillede ((??? IKKE OGSÅ???)).
Det påhviler kommunerne at tildele støtte efter servicelovens bestemmelser, som beskrevet i fx §??? ((UDDRAG?? NÆPPE PLADS)).
Kommunerne er dog underlagt flere fordringer end, at sikre de handicappedes rettigheder, som bl.a. 'vejleding til støtte til voksne' understreger. (KL og socialministeriet - Danske Regioner??)
Her understreges, at der tildeles konkret og tilrettelagt støtte efter individets faktiske behov.
En given diagnose skal ikke 'garantere' en specifik intervention. ((CHECK KILDE))
Set i lyset af en ressourcefordelingsparadigme, kan dette synes rimeligt nok. Dette underbygger og understreger dog 

\subsection{De professionelle}
((FOR MEGET HISTORIE TROR JEG))
Den (velfærds)profession, der forbindes med arbejdet med handicappede\footnote{Handicappede indgår i, hvad der skiftevis det specialiserede socialområde eller det specialiserede voksenomráde}, er pædagogerne.

\citeauthor{goffmanAsylumsEssaysSocial1991} siger i \citetitle{goffmanAsylumsEssaysSocial1991}, at de ansatte  i institutionen tilegner sig en særlig menneskeforståelse, på baggrund af den totale institutions selvfortellinger om institutionens formål og berættigelse.
Dog er der (forhåbentlig) lang tid siden 1950ernes store institutioner; men man kan se ud fra SLs værdigrundlag at professionens fagforening ser kerneopgaven som at være:

\subsection{Problemforståelsen i dansk handicappolitik i dag}
Det ovenstående peger på, en forskydning af \textit{problemet der skal løses} omkring de handicapped.
Fra at være til gene for omgivelserne, med en medicinsk afvigelsesforståelse (åndssvageanstalernes forstandere var læger) over mennesker at aktivt skulle intregreres og socialiseres, til \textit{individer} med rettigheder — og pligter — i og overfor samfundet.
Med afsæt i \citeauthor{scheurichPolicyArchaeologyNew1994}, kan man udforske, hvordan vi er nået hertil.
