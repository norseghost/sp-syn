\section{Handicappolitik i Danmark}

Jeg formoder, at jeg vil tage fat i den specialpædagogiske professions selvforståelse i mit speciale.
'Specialpædagogik' er forbundet med arbejdet med mennesker med \textit{nedsat fysisk eller kognitiv funktionsevne}, som det så fint hedder i \citetitle{social-ogindenrigsministerietBekendtgorelseAfLov2019} — eller \textit{handicappede}, om man vil. Jeg vil forholdsvis konsekvent holde mig til 'handicappede' fremfor den mere forsigtige formulering ovenfor fremadrettet.

I den følgende synopsis vil jeg diskutere nedslag i en policy-analyse af dansk handicappolitik, som den fremstår i dag.
Analysens formål er, en forståelse af, hvordan 'de handicappede' er blevet et samfundsmæssigt anliggende, og hvordan dette anliggende imødekommes.
Dermed søger jeg i min undersøgelse svar på følgende problemformulering:

Hvordan fremstår problemopfattelsen i dansk handicappolitik i dag; og hvilke løsnigsforslag legitimeres, hendholdsvis delegitimeres, på baggrund af dette?

Min undersøgelse er på baggrund af \citetitle{klStyringAfDet2017}, en vejledning til det specialiserede voksenområde udarbejdet af Kommunernes Landsforening i samarbejde med Børne- og socialministeriet, Økonomi- og indenrigsministeriet samt Finansministeriet \autocite{klStyringAfDet2017}.
Undervejs vil jeg henvise til relevante afsnit i \citetitle{social-ogindenrigsministerietBekendtgorelseAfLov2019}, da det er heri, rammerne for den danske socialpolitik (og dermed også handicapolitik) fastlægges, som kommunerne skal udføre i praksis \autocite[kapitel 2]{social-ogindenrigsministerietBekendtgorelseAfLov2019}. 
Metodisk er analysen bygget op med afsæt i \citetitle{scheurichPolicyArchaeologyNew1994}, af \citeauthor{scheurichPolicyArchaeologyNew1994}.

\citeauthor{scheurichPolicyArchaeologyNew1994} anfægter selvfølgeligheden af de “sociale problemer” som et empirisk fænomen \autocite[s. ??]{scheurichPolicyArchaeologyNew1994}.
I stedet for bør og skal de såkaldte “sociale problemer” undersøges som de sociale konstruktioner de er.
\citeauthor{scheurichPolicyArchaeologyNew1994} præsenterer policy-arkæologien som en alternativ analysemodel til de (post)positiviske analysemodeller, der, ved at tage for givet, at noget er et \textbf{problem}, også forudsætter, at der er en \textbf{løsning}, der kan sikre samfundet stabilitet og videreførelse.
Herved kan man opnå en policy-analyse der går ud over det historiske, evaluerende, og/eller komparative.

Policy-arkæologien præsenteres med fire undersøgelsesfelter \autocite[s. 300]{scheurichPolicyArchaeologyNew1994}:
\begin{enumerate}
  \item
    den sociale konstruktion af specifikke sociale problemer
  \item
    identifikation af sociale regelmæssigheder på tværs af problemdefinitioner
  \item
    den sociale konstruktion af acceptable løsningsmuligheder
  \item
    policy-analysens sociale formål
\end{enumerate}

Jeg vil i det følgende komme med bud på, hvordan disse undersøgelsesfelter kan besvares, med \citetitle{klStyringAfDet2017} som undersøgelsesobjekt.
Denne pjece er udarbejdet for at \textit{afdække styringsudfordringer og -muligheder} på det specialiserede socialområde\footnote{Der benyttes skiftevis “voksenområde” og “socialområde” i pjecen.} \autocite[s. 2]{klStyringAfDet2017}, der bland andet omfatter voksne med handicap.\todo{fodnode omhandlende børn og unge?}
Der omtales i særdeleshed et "præs” i pjecen - “udgiftspræs”, “styringspræs”; og formålet med udgivelsen er, at bidrage med konkrete "styringsværktøjer” til at “påvirke udgiftsniveauet” \autocite[s. 2, 3. m.fl.]{klStyringAfDet2017}.
Derfor mener jeg, at se et eksempel på anvendt policy — konkrete handlingsanvisninger ud fra rammerne i \citetitle{social-ogindenrigsministerietBekendtgorelseAfLov2019}.

\section{Handicappede i Danmark — eller: konstruktionen af et socialt problem}
Der er en (overvejende) relationel forankring i handicapbegrebet i dansk sammenhæng.
I en sådan handicapforståelse, er \textit{individet} ikke handicappet i sig selv. Det er i relationen mellem funktionsnedsættelser og mulige barrierer i samfundet,  at der opstår et handicap \autocite[s. 41]{alma99122456944705763}.
\footnote{Dette ses typisk i kontrast til en medicinsk handicapforståelse, hvor individet er i besiddelse af skavanker, der er afvigelser fra normen \autocite[s. 37]{alma99122456944705763}}.

Handicap\textit{politikken} i dag er derudover præget af en normaliserings- og individualiseringslogik, der blev tydeliggjort i 1998, hvor \citetitle{social-ogindenrigsministerietBekendtgorelseAfLov2019} blev vedtaget.\todo{Kilde}

\subsection{Problemforståelsen som den fremstår i pjecen}\todo{holdes kort - problemforståelsen kan måske bedst illustreres via de acceptable løsningsforslag?}
Denne forskydning af \textit{problemet der skal løses} omkring de handicappede kan illustreres med følgende udpluk fra pjecen: \todo{parafraser eller meget korte udpluk}
\blockquote{visitationen tager udgangspunkt i borgerens behov samtidig med, at der sikres generel budgetoverholdelse} (s. 18)
\blockquote{Det giver borgeren mulighed for at leve et så selvstændigt og meningsfuldt liv som muligt med mindst mulig eller ingen støtte.}(s. 19)
\section{De sociale regelmæssigheder}

\citeauthor{scheurichPolicyArchaeologyNew1994} mener, at man kan se regelmæssigheder på tværs af sociale problemer.
Disse regelmæssigheder er med til at synliggøre noget som et “socialt fænomen”; samtidig med, at det også udgør de socialt troværdige løsningsmuligheder \cite[s. 301]{scheurichPolicyArchaeologyNew1994}
Hvilke regelmæssighed - det vil sige, forbindelser til det omkringliggende samfund — kan vi finde i de forskellige perspektiver nævnt ovenpå?

\begin{itemize}
  \item
    et individorienteret fokus, 
  \item
    hvor frihed udspringer af rettigheder
  \item
    men der er en medfølgende ansvarsfordring
  \item
    om at være produktiv
  \item
    og man skal have den \textit{påkrævede} støtte, hverken mere eller mindre
\end{itemize}\todo{omskrives/slettes}


Her kan man trække paralleller ud til resten af det \textit{specialiserede voksenomráde}, der ud over handicap, også omfatter \quote{udsatte børn og unge} eller mennesker med \quote{særlige sociale problemer} \autocite[s. 2]{kommunerneslandsforeningFaktaOmKommnernes2019}.
Indsatserne her drejer sig gerne om hjælp til selvhjælp, med fokus på tidlige og forebyggende indsatser — for at undgå et mere omfattende (og dermed mere omkostningsfyldt) støttebehov i fremtiden. Der tales også i vid omstrækning om “den helhedsorienterede tilgang", for at forhindre, at der spildes ressourcer i sideløbende sociale indsatser. \autocite[s. 18]{klStyringAfDet2017}

\section{Individet i fokus — eller: konstruktionen af acceptable løsningsforslag}
Der præciseres, at individet skal have, den støtte der er brug for.
Dermed må og skal støtten ændre karakter, hvis hjælpebehovet ændrer sig.
Dette kan fx medføre, at man skal opgive sit hjem, hvis der vurderes, at man ikke får imødekoomet sit støttebehov hvor man bor.
\blockquote{understøtte borgerens progression og på at udvikle, styrke og bevare borgerens egne ressourcer}
\blockquote{Kommunerne har de senere år været igennem en omstilling, hvor en rehabiliterende tilgang i vidt omfang er blevet udgangspunktet for de sociale indsatser. Med den rehabiliterende tilgang sættes der fokus på at understøtte borgerens progression og på at udvikle, styrke og bevare borgerens egne ressourcer og dermed støtte borgeren i at opretholde en selv- stændig tilværelse uden behov for hjælp i videst muligt omfang.}\autocite[s. 19]{klStyringAfDet2017}

fremskudt rådgivning/din indgang cases s. 20 -> afklaring af støttebehov; der samtidig kan fange de små problemer inden de bliver store (dyre) problemer

\textit{flow side 19 -> det følger af rehabiliteringen?} vender tilbage s. 22; sammen med indsatstrappen

indsatstrappen fordrer opfølgning; har behovet ændret sig, kan den visiterede støtte det også

\section{Hvad (og hvem) gør denne analyse godt for?}
Jeg forsøger at følge i \citetitle{scheurichPolicyArchaeologyNew1994} sine fodspor, og stille et kritisk blik på dansk handicappolitik.
Analysen der drøftes her, kan nærmest anses for, at være en meta-analyse; idet analyseobjektet har et evaluerende blik på socialpolitiske interventioner.
Dog falder \citetitle{klStyringAfDet2017} i netop den “fælde”, som \citeauthor{scheurichPolicyArchaeologyNew1994} beskriver: man tager problemet for givet (her spændingsfeltet mellem borgernes rettigheder efter loven og kommunalbudgetterne); og søger en løsning, der kan opretholde status quo.
