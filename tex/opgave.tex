\section{Handicappolitik i Danmark}

Jeg formoder, at jeg vil tage fat i den specialpædagogiske professions selvforståelse i mit speciale.
'Specialpædagogik' er forbundet med arbejdet med mennesker med \textit{nedsat fysisk eller kognitiv funktionsevne}, som det hedder i \citetitle{social-ogindenrigsministerietBekendtgorelseAfLov2019} — eller \textit{handicappede}, om man vil. Jeg vil forholdsvis konsekvent holde mig til “handicappede” fremadrettet.

I den følgende synopsis vil jeg diskutere nedslag i en policy-analyse af dansk handicappolitik, som den fremstår i dag.
Analysens formål er, en forståelse af, hvordan “de handicappede” ses som samfundsmæssigt anliggende, og hvordan dette anliggende imødekommes.
Dermed søger jeg i min undersøgelse svar på følgende problemformulering:
\begin{center}
  \textit{Hvordan fremstår problemopfattelsen i dansk handicappolitik i dag; og hvilke løsningsforslag legitimeres, henholdsvis delegitimeres, på baggrund af dette?}
\end{center}
Min undersøgelse er på baggrund af \citetitle{social-ogindenrigsministerietBekendtgorelseAfLov2019}, da det er heri, rammerne for den danske socialpolitik (og dermed også handicapolitik) fastlægges \autocite[kapitel 2]{social-ogindenrigsministerietBekendtgorelseAfLov2019}, samt \citetitle{klStyringAfDet2017}\footnote{Analysen, ligesom undersøgelsesobjektet, forholder sig ikke til indsatser målrettet børn og unge.}
, en vejledning til det specialiserede voksenområde \autocite{klStyringAfDet2017}.
Metodisk er analysen bygget op med afsæt i policy-arkæologien, som fremlagt af \citeauthor{scheurichPolicyArchaeologyNew1994}, \citeyear{scheurichPolicyArchaeologyNew1994}.

\citeauthor{scheurichPolicyArchaeologyNew1994} \todo{Kilde} anfægter selvfølgeligheden af de “sociale problemer” som et empirisk fænomen \autocite[s. ??]{scheurichPolicyArchaeologyNew1994}.
I stedet for bør og skal de såkaldte “sociale problemer” undersøges som de sociale konstruktioner de er.
\citeauthor{scheurichPolicyArchaeologyNew1994} kritiserer de (post)positivistiske analysemodeller, der, ved at tage for givet, at noget er et \textbf{problem}, også forudsætter, at der kan findes en \textbf{løsning}, der kan sikre samfundet stabilitet og videreførelse.\todo{Kilde}
Policy-arkæologien præsenteres som en alternativ analysemodel, der går ud over det historiske, evaluerende, og/eller komparative.

Jeg vil disponere det følgende efter policy-arkæologiens fire undersøgelsesfelter \autocite[s. 300]{scheurichPolicyArchaeologyNew1994}:
\begin{enumerate}
  \item
    den sociale konstruktion af specifikke sociale problemer - med afsæt i \citetitle{social-ogindenrigsministerietBekendtgorelseAfLov2019}
  \item
    identifikation af sociale regelmæssigheder på tværs af problemdefinitioner
  \item
    den sociale konstruktion af acceptable løsningsmuligheder - med udgangspunkt i \citetitle{klStyringAfDet2017}.
  \item
    policy-analysens sociale formål
\end{enumerate}

\citetitle{klStyringAfDet2017} er udarbejdet for at \textit{afdække styringsudfordringer og -muligheder} på det specialiserede socialområde\footnote{Der benyttes skiftevis “voksenområde” og “socialområde” i pjecen.} \autocite[s. 2]{klStyringAfDet2017}, der bland andet omfatter voksne med handicap.
Der omtales i særdeleshed et "pres” i pjecen - “udgiftspres”, “styringspres”; og formålet med udgivelsen er, at bidrage med konkrete "styringsværktøjer” til at “påvirke udgiftsniveauet” \autocite[s. 2, 3. m.fl.]{klStyringAfDet2017}.
Derfor mener jeg, at se et eksempel på anvendt policy — konkrete handlingsanvisninger ud fra rammerne i \citetitle{social-ogindenrigsministerietBekendtgorelseAfLov2019}.

\section{Konstruktionen af et socialt problem, som den fremstår i \citetitle{social-ogindenrigsministerietBekendtgorelseAfLov2019}}

Allerede i indledningen gøres det klart, at hjælpen der tilbydes skal fremme individets udvikling og selvstændighed; og derudover øge livskvaliteten.
Hjælpen er videre funderet i et individuelt ansvar til udvikling og udnyttelse af potentialer, og skal tilrettelægges efter en “konkret vurdering” af behovet.
Det individuelle ansvar kommer også til syne, idet den hjælpetrængende skal inddrages i et samarbejde om støtten der gives \autocite[§§ 2-3]{social-ogindenrigsministerietBekendtgorelseAfLov2019}.

Også i formålsbeskrivelsen til støtten der gives på det specialiserede voksenområde fremgår  dette individorienterede fokus.
Udover individets styrkede muligheder og ansvar for udvikling, er der også en normaliseringsorientering mod deltagelse i samfundet.
Der skal derudover sikres \textit{en helhedsorienteret støtte} \autocite[§ 81]{social-ogindenrigsministerietBekendtgorelseAfLov2019}.

Dermed tegner det sig et billede af et problem: Der er individer i samfundet der ikke opfylder de krav og forpligtelser om selvstændighed og ansvar for eget liv, deres rettigheder forudsætter. Dette skal man så finde en løsning for i kommunerne.

Ud fra pjecen kan derudover læse, at den anden side af \textbf{problemet} er, at disse opgaver skal løses indenfor kommunalbudgetternes økonomiske ramme.

\section{De sociale regelmæssigheder}

\citeauthor{scheurichPolicyArchaeologyNew1994} mener, at man kan se regelmæssigheder på tværs af sociale problemer.
Disse regelmæssigheder er med til at synliggøre noget som et “socialt fænomen”; samtidig med, at det også udgør de socialt troværdige løsningsmuligheder \autocite[s. 301]{scheurichPolicyArchaeologyNew1994}
Hvilke regelmæssigheder - det vil sige, forbindelser til det omkringliggende samfund — kan vi finde i de forskellige perspektiver nævnt ovenfor?
Her kan man trække paralleller ud til resten af teksten i \citetitle{social-ogindenrigsministerietBekendtgorelseAfLov2019} .
Indsatserne overfor andre målgrupper — fx børn og unge med særlige behov herfor, voksne med særlige sociale problemer — drejer sig gerne om hjælp til selvhjælp, med fokus på tidlige og forebyggende indsatser. Der tales også i vid omstrækning om “den helhedsorienterede tilgang", for at forhindre, at der spildes ressourcer i sideløbende sociale indsatser. \autocite[Kap. 6, 15]{social-ogindenrigsministerietBekendtgorelseAfLov2019}.

Dermed folder handicapindsatserne sig ind i den bredere socialpolitik, med:
\begin{itemize}
  \item
    et individorienteret fokus, 
  \item
    hvor frihed udspringer af rettigheder
  \item
    men der er en medfølgende ansvarsfordring
  \item
    og man skal have den \textit{påkrævede} støtte, hverken mere eller mindre
\end{itemize}
\section{Individet i fokus — eller: konstruktionen af mulige løsningsforslag}
\todo{for langt}
Ved at se på de cases, der presenteres i pjecens kapitel 3 har man eksempler på anbefalinger til løsningsforslag.
Men hvorfor og hvordan er disse særligt legitime? 

Med stikord som “understøtte borgerens progression” og “styrke ... borgerens egne ressourcer” og “opretholde en selvstændig tilværelse” \autocite[s. 19]{klStyringAfDet2017} forstærkes og legitimeres det individfokus, der er i \citetitle{social-ogindenrigsministerietBekendtgorelseAfLov2019} yderligere.

Derudover er der fokus på \textit{visitationen}, og \textit{udredningen}, hvor man kan give borgeren “den bedste og billigste indsats” der “tager udgangspunkt i borgerens behov samtidig med, at der sikres generel budgetoverholdelse” \autocite[ss. 18, 28-30]{klStyringAfDet2017}

Den helhedsorienterede tilgang anbefales også, for at undgå parallelle og modsatrettede forløb for borgeren \autocite[ss. 34-35]{klStyringAfDet2017}

Der er også cases der sætter en tidlig indsats i fokus.
Hermed kan man afklare den enkeltes støttebehov, og samtidig fange de små problemer inden de bliver store (og dyre) problemer\autocite[ss 20-21]{klStyringAfDet2017}.
Til at opfylde det påkrævede støttebehov henvises til indsatstrappen; hvor man kan bevæge op og ned af indsatstrin.
Dette fordrer dog opfølgning; har behovet ændret sig, kan den visiterede støtte det også.

Det økonomiske skøn bedes også inddrages. Her understreges der, at kommunen træffer afgørelsen om støtte — og at borgeren ikke kan “bestille” en specifik indsats [s. 31].
I det \citetitle{social-ogindenrigsministerietBekendtgorelseAfLov2019} er en rammelov, kan den økonomiske side af problemet også imødekommes vha. budgetmodeller og socialt serviceniveau \autocite[ss. 36-37]{klStyringAfDet2017}.

\section{Hvad (og hvem) gør denne analyse godt for?}
Jeg forsøger at følge i \citetitle{scheurichPolicyArchaeologyNew1994} sine fodspor, og se kritisk på dansk handicappolitik.
Analysen der drøftes her, kan nærmest anses for, at være en meta-analyse; idet analyseobjektet har et evaluerende blik på socialpolitiske interventioner.
Dog falder \citetitle{klStyringAfDet2017} i netop den “fælde”, som \citeauthor{scheurichPolicyArchaeologyNew1994} beskriver: man tager problemet for givet. (for pjecens vedkommende spændingsfeltet mellem borgernes rettigheder efter loven og kommunalbudgetterne; der i øvrigt expliciteres i \citetitle[§ 1, stk 3]{social-ogindenrigsministerietBekendtgorelseAfLov2019}); og søger en løsning, der kan opretholde status quo.
De legitime løsninger springer direkte ud af problemforståelsen; der også fremgår af lov.
\todo{bedre afslutning}
