\section{Handicappolitik i Danmark}

Jeg formoder, at jeg vil tage fat i den specialpædagogiske professions selvforståelse i mit speciale.
'Specialpædagogik' er forbundet med arbejdet med mennesker med \textit{nedsat fysisk eller kognitiv funktionsevne}, som det så fint hedder i \citetitle{social-ogindenrigsministerietBekendtgorelseAfLov2019} — eller \textit{handicappede}, om man vil. Jeg vil forholdsvis konsekvent holde mig til 'handicappede' fremfor den mere forsigtige formulering ovenfor fremadrettet.

I den følgende synopsis vil jeg diskutere nedslag i en policy-analyse af dansk handicappolitik, som den fremstår i dag.
Analysens formål er, en forståelse af, hvordan 'de handicappede' er blevet et samfundsmæssigt anliggende, og hvordan dette anliggende imødekommes.
Dermed søger jeg i min undersøgelse svar på følgende problemformulering:

Hvordan fremstår problemopfattelsen i dansk handicappolitik i dag; og hvilke løsnigsforslag legitimeres, hendholdsvis delegitimeres, på baggrund af dette?

Min undersøgelse er på baggrund af \citetitle{klStyringAfDet2017}, en vejledning til det specialiserede voksenområde udarbejdet af Kommunernes Landssammenslutning i samarbejde med Børne- og socialministeriet, Økonomi- og indenrigsministeriet samt Finansministeriet \autocite{klStyringAfDet2017}.
Undervejs vil jeg henvise til relevante afsnit i \citetitle{social-ogindenrigsministerietBekendtgorelseAfLov2019}, da det er heri, rammerne for den danske handicapolitik fastlægges \todo{find pragraf til eksempel}\autocite[§ ??]{social-ogindenrigsministerietBekendtgorelseAfLov2019}. 
Metodisk er analysen bygget op med afsæt i \citetitle{scheurichPolicyArchaeologyNew1994}, af \citeauthor{scheurichPolicyArchaeologyNew1994}.

\citeauthor{scheurichPolicyArchaeologyNew1994} anfægter selvfølgeligheden af de “sociale problemer” som et empirisk fænomen \autocite[s. ??]{scheurichPolicyArchaeologyNew1994}.
I stedet for bør og skal de såkaldte “sociale problemer” undersøges som de sociale konstruktioner de er.
\citeauthor{scheurichPolicyArchaeologyNew1994} præsenterer policy-arkæologien som en alternativ analysemodel til de (post)positiviske analysemodeller, der, ved at tage for givet, at noget er et \textbf{problem}, også forudsætter, at der er en \textbf{løsning}, der kan sikre samfundet stabilitet og videreførelse.
Herved kan man opnå en policy-analyse der går ud over det historiske, evaluerende, og/eller komparative.

Policy-arkæologien præsenteres med fire undersøgelsesfelter \autocite[s. 300]{scheurichPolicyArchaeologyNew1994}:
\begin{enumerate}
  \item
    den sociale konstruktion af specifikke sociale problemer
  \item
    identifikation af sociale regelmæssigheder på tværs af problemdefinitioner
  \item
    den sociale konstruktion af acceptable løsningsmuligheder
  \item
    policy-analysens sociale formål
\end{enumerate}

Jeg vil i det følgende komme med bud på, hvordan disse undersøgelsesfelter kan besvares, med \citetitle{klStyringAfDet2017} som undersøgelsesobjekt. 

\section{Handicappede i Danmark — eller: konstruktionen af et socialt problem}
Der er en (overvejende) relationel forankring i handicapbegrebet i dansk sammenhæng.
I følge en relationel handicapforståelse, er \textit{individet} ikke handicappet - men bliver det, i det omfang, omgivelserne ikke tager højde for eventuelle funktionsnedsættelser \todo{KILDE}.
\footnote{Dette ses typisk i kontrast til en medicinsk handicapforståelse\todo{KILDE}, hvor individet er i besiddelse af skavanker, der er afvigelser fra normen; eller endda anses for, at være afvigende i sig selv.}

Men det har ikke altid forholdt sig sådan.
Vi er gået fra en medicinsk afvigelsesforståelse op igennem 60erne (åndssvageanstalernes forstandere var læger);\todo{kilde} over mennesker, der aktivt skulle integreres og socialiseres i 70erne og 80erne, til \textit{individer} med rettigheder — og pligter — i og overfor samfundet fra 90erne og frem til i dag.
Handicap\textit{politikken} i dag er præget af den normaliserings- og individualiseringslogik,  der blev tydeliggjort i 1998, hvor \citetitle{social-ogindenrigsministerietBekendtgorelseAfLov2019} blev vedtaget.\todo{Kilde}

\subsection{Problemforståelsen i dansk handicappolitik i dag}
Denne forskydning af \textit{problemet der skal løses} omkring de handicappede kan illustreres med følgende udpluk af....

\section{De sociale regelmæssigheder}

\citeauthor{scheurichPolicyArchaeologyNew1994} mener, at man kan se regelmæssigheder på tværs af sociale problemer.
Disse regelmæssigheder er med til at synliggøre noget som et “socialt fænomen”; samtidig med, at det også udgør de socialt troværdige løsningsmuligheder \cite[s. 301]{scheurichPolicyArchaeologyNew1994}
Hvilke regelmæssighed - det vil sige, forbindelser til det omkringliggende samfund — kan vi finde i de forskellige perspektiver nævnt ovenpå?

\begin{itemize}
  \item
    et individorienteret fokus, 
  \item
    hvor frihed udspringer af rettigheder
  \item
    men der er en medfølgende ansvarsfordring
  \item
    om at være produktiv
  \item
    og man skal have den \textit{påkrævede} støtte, hverken mere eller mindre
\end{itemize}

Her kan man trække paralleller ud til resten af det \textit{specialiserede voksenomráde}, der ud over handicap, også rummer psykiatri, stofmisbrug...\todo{og hvad mere?}.
Indsatserne her drejer sig gerne om hjælp til selvhjælp, med fokus på tidlige og forebyggende indsatser — for at undgå et mere omfattende (og dermed mere omkostningsfyldt) støttebehov i fremtiden.

\section{Individet i fokus — eller: konstruktionen af acceptable løsningsforslag}
Der præciseres, at individet skal have, den støtte der er brug for.
Dermed må og skal støtten ændre karakter, hvis hjælpebehovet ændrer sig.
Dette kan fx medføre, at man skal opgive sit hjem, hvis der vurderes, at man ikke får imødekoomet sit støttebehov hvor man bor. 

\section{Hvad (og hvem) gør denne analyse godt for?}
Jeg forsøger at følge i \citetitle{scheurichPolicyArchaeologyNew1994} sine fodspor, og stille et kritisk blik på dansk handicappolitik.
Analysen der drøftes her, kan nærmest anses for, at være en meta-analyse; idet analyseobjektet har et evaluerende blik på socialpolitiske interventioner.
