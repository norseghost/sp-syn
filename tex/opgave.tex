\section{Indledning}

Som så meget moderne velfærdsarbejde, fremstår også arbejdet med mennesker med \textbf{nedsat fysisk eller kognitiv funktionsevne}, som det hedder i \citetitle{social-ogindenrigsministerietBekendtgorelseAfLov2019} — eller \textbf{handicappede}, om man vil. \footnote{Jeg vil holde mig til “handicappede” fremadrettet.} fyldt af paradokser, hvor pædagogerne skal navigere modsatrettede og flertydige krav \autocite{mik-meyerIndledningSkabeProfessionel2012}.

For at bedre forstå disse paradokser, vil jeg diskutere nedslag i en policy-analyse af dansk handicappolitik, som den fremstår i dag.
Analysens formål er, en forståelse af, hvordan “de handicappede” ses som samfundsmæssigt anliggende, og hvordan dette anliggende imødekommes.
Dermed søger jeg i min undersøgelse svar på følgende problemformulering:

Hvordan fremstår problemopfattelsen i dansk handicappolitik i dag; og hvilke løsningsforslag legitimeres, henholdsvis delegitimeres, på baggrund af dette?

Min undersøgelse er på baggrund af \citetitle{social-ogindenrigsministerietBekendtgorelseAfLov2019}, da det er heri, rammerne for den danske socialpolitik (og dermed også handicapolitik) fastlægges \autocite[kapitel 2]{social-ogindenrigsministerietBekendtgorelseAfLov2019}, samt \citetitle{klStyringAfDet2017}\footnote{Analysen, ligesom pjecen, forholder sig ikke til indsatser målrettet børn og unge.}, en vejledning til det specialiserede voksenområde \autocite{klStyringAfDet2017}.
Metodisk er analysen bygget op med afsæt i policy-arkæologien, som fremlagt af \citeauthor{scheurichPolicyArchaeologyNew1994}, \citeyear{scheurichPolicyArchaeologyNew1994}.

\citeauthor{scheurichPolicyArchaeologyNew1994} anfægter selvfølgeligheden af de “sociale problemer” som et empirisk fænomen.
I stedet for bør og skal de såkaldte “sociale problemer” undersøges som de sociale konstruktioner de er.
\citeauthor{scheurichPolicyArchaeologyNew1994} kritiserer de (post)positivistiske analysemodeller, der, ved at tage for givet, at noget er et \textbf{problem}, også forudsætter, at der kan findes en \textbf{løsning}, der kan sikre samfundet stabilitet og videreførelse. \autocite[ss. 298-299]{scheurichPolicyArchaeologyNew1994}.

Policy-arkæologien præsenteres med fire undersøgelsesfelter \autocite[s. 300]{scheurichPolicyArchaeologyNew1994}:
\begin{enumerate}
  \item
    den sociale konstruktion af specifikke sociale problemer - med afsæt i \citetitle{social-ogindenrigsministerietBekendtgorelseAfLov2019}
  \item
    identifikation af sociale regelmæssigheder på tværs af problemdefinitioner
  \item
    den sociale konstruktion af acceptable løsningsmuligheder - med udgangspunkt i \citetitle{klStyringAfDet2017}.
  \item
    policy-analysens sociale formål
\end{enumerate}

Jeg vil i det følgende drøfte felterne 1, 3 og 4.\footnote{De sociale regelmæssigheder vender jeg tilbage til ved eksaminationen.}

\citetitle{klStyringAfDet2017} er udarbejdet for at \textbf{afdække styringsudfordringer og -muligheder} på det specialiserede voksenområde.
Der omtales i særdeleshed et "pres” i pjecen - “udgiftspres”, “styringspres”; og formålet med udgivelsen er, at bidrage med konkrete "styringsværktøjer” til at “påvirke udgiftsniveauet” \autocite[s. 2, 3. m.fl.]{klStyringAfDet2017}.
Derfor mener jeg, at se et eksempel på anvendt policy — konkrete handlingsanvisninger ud fra rammerne i \citetitle{social-ogindenrigsministerietBekendtgorelseAfLov2019}.

\section{Konstruktionen af et socialt problem, som den fremstår i \citetitle{social-ogindenrigsministerietBekendtgorelseAfLov2019}}

Allerede i indledningen gøres det klart, at hjælpen der tilbydes skal fremme individets udvikling og selvstændighed; og derudover øge livskvaliteten.
Hjælpen er videre funderet i et individuelt ansvar til udvikling og udnyttelse af potentialer; og i et samarbejde med borgeren \autocite[§§ 2-3]{social-ogindenrigsministerietBekendtgorelseAfLov2019}.

Også i formålsbeskrivelsen til støtten der gives på det specialiserede voksenområde fremgår  dette individorienterede fokus.
Udover individets styrkede muligheder og ansvar for udvikling, er der også en normaliseringsorientering mod deltagelse i samfundet \autocite[§ 81]{social-ogindenrigsministerietBekendtgorelseAfLov2019}.

Dermed tegner det sig et billede af et problem: Der er individer i samfundet der ikke opfylder de krav og forpligtelser om selvstændighed og ansvar for eget liv, deres rettigheder forudsætter.

\citeauthor{foucaultOvervagningOgStraf2005} omtaler magtens produktive potentialer, idet magthavernes behov for kontrol skabte nogen at kontrollere --- subjekter, individer, der socialiseres til selvkontrol \autocite{foucaultOvervagningOgStraf2005}.
Problemforståelsen set i dette perspektiv kunne være, at tilskynde evnen til selvstyre og selvkontrol.
Men så skal det produktive potentiale der ligger i den pædagogiske \textbf{magt}relation komme til udtryk; og ikke undgås \autocite{hurFrigorelsensMagt2015}.
Her er der en risiko for blinde pletter; hvor fokus på borgerens integritet læses som en fordring til “ikkehandling”, alt for at undgå krænkelser og overgreb \autocite{langagerDetAfmalteLiv2013}.

\section{De anbefalede løsningsforslag}
Ved at se på de cases, der præsenteres i pjecens kapitel 3 har man eksempler på anbefalinger til løsningsforslag \autocite[ss. 18-40]{klStyringAfDet2017}.

Med stikord som “styrke ... borgerens egne ressourcer” og “opretholde en selvstændig tilværelse” forstærkes og legitimeres individfokuset yderligere.

Det økonomiske skøn bedes også inddrages. Her understreges der, at kommunen træffer afgørelsen om støtte — og at borgeren ikke kan “bestille” en specifik indsats. Det \textbf{økonomiske} hensyn kan dermed overtrumfe borgerens rolle som “ekspert i eget liv”; der underkendes yderligere af fokuset på \textbf{visitationen} og \textbf{udredningen}; hvor man kan give borgeren “den bedste og billigste indsats” der “tager udgangspunkt i borgerens behov samtidig med, at der sikres generel budgetoverholdelse” .

Den \textbf{tidlige indsats} kan afklare den enkeltes støttebehov, og samtidig fange de små problemer inden de bliver store (og dyre) problemer.

Og det er vigtigt med \textbf{opfølgning}; har behovet ændret sig, kan den visiterede støtte det også.

Men hvorfor og hvordan er dette særligt legitime, anbefalede løsninger? 

Det nemme svar er, at 'det står i loven' som kommunerne er forpligtede på.
En mere dybdegående analyse ser repræsentationer af socialt imaginære betydningsdannelser  \autocite[s. 91]{moutsiosSocialeInstitutionerOg2016} omkring (blandt andet) autonomi, værdighed og økonomisk mådehold --- der nogle gange trækker i forskellige retninger.
Man kan ane en særlig “teori om menneskets natur” lignende hvad, der beskrives af \citeauthor{goffmanAsylumsEssaysSocial1991} blandt de ansatte på institutionen \autocite[ss. 84-85]{goffmanAsylumsEssaysSocial1991}, der dog nogle gange kommer til kort, når det er tale om borgere der ikke har forudsætninger for, at kunne leve op til sådanne forventninger og forpligtelser.
Jeg mener, man som pædagog skal anerkende, når man tager styringen i en andens liv \autocite{andersenSocialWorkPower2018}.
Dette møder dog modstand - den edukative diskurs, der netop fordrer menneskets \textbf{selv}humanisering, er også udbredt til de handicappede \autocite[s. 77]{lieberkindUddannelsessamfundetOgEdukative2016}.

\section{Pjecens sociale formål}
Jeg forsøger at gøre \citeauthor{scheurichPolicyArchaeologyNew1994} efter, og se kritisk på dansk handicappolitik.
Analysen der drøftes her, kan anses for, at være en meta-analyse; idet \citetitle{klStyringAfDet2017} har et evaluerende blik på socialpolitiske interventioner.
Pjecen falder i netop den “fælde”, som \citeauthor{scheurichPolicyArchaeologyNew1994} beskriver: man tager problemet for givet, og søger en løsning, der kan opretholde samfundets sammenhængskraft.

Dette legitimerer og reproducerer de sociale regelmæssigheder der konstituerer problemet såvel som løsningerne som “virkelige”. Dermed socialiseres velfærdens håndhævere i, at det individuelle fokus både er bedst for kommunalbudgetterne såvel som den enkelte.
